\documentclass[]{article}
\usepackage{lmodern}
\usepackage{amssymb,amsmath}
\usepackage{ifxetex,ifluatex}
\usepackage{fixltx2e} % provides \textsubscript
\ifnum 0\ifxetex 1\fi\ifluatex 1\fi=0 % if pdftex
  \usepackage[T1]{fontenc}
  \usepackage[utf8]{inputenc}
\else % if luatex or xelatex
  \ifxetex
    \usepackage{mathspec}
  \else
    \usepackage{fontspec}
  \fi
  \defaultfontfeatures{Ligatures=TeX,Scale=MatchLowercase}
\fi
% use upquote if available, for straight quotes in verbatim environments
\IfFileExists{upquote.sty}{\usepackage{upquote}}{}
% use microtype if available
\IfFileExists{microtype.sty}{%
\usepackage{microtype}
\UseMicrotypeSet[protrusion]{basicmath} % disable protrusion for tt fonts
}{}
\usepackage[margin=1in]{geometry}
\usepackage{hyperref}
\hypersetup{unicode=true,
            pdftitle={Final Assignment: Predicting Director Compensation},
            pdfauthor={Jasper Ginn (s6100848)},
            pdfborder={0 0 0},
            breaklinks=true}
\urlstyle{same}  % don't use monospace font for urls
\usepackage{graphicx,grffile}
\makeatletter
\def\maxwidth{\ifdim\Gin@nat@width>\linewidth\linewidth\else\Gin@nat@width\fi}
\def\maxheight{\ifdim\Gin@nat@height>\textheight\textheight\else\Gin@nat@height\fi}
\makeatother
% Scale images if necessary, so that they will not overflow the page
% margins by default, and it is still possible to overwrite the defaults
% using explicit options in \includegraphics[width, height, ...]{}
\setkeys{Gin}{width=\maxwidth,height=\maxheight,keepaspectratio}
\IfFileExists{parskip.sty}{%
\usepackage{parskip}
}{% else
\setlength{\parindent}{0pt}
\setlength{\parskip}{6pt plus 2pt minus 1pt}
}
\setlength{\emergencystretch}{3em}  % prevent overfull lines
\providecommand{\tightlist}{%
  \setlength{\itemsep}{0pt}\setlength{\parskip}{0pt}}
\setcounter{secnumdepth}{0}
% Redefines (sub)paragraphs to behave more like sections
\ifx\paragraph\undefined\else
\let\oldparagraph\paragraph
\renewcommand{\paragraph}[1]{\oldparagraph{#1}\mbox{}}
\fi
\ifx\subparagraph\undefined\else
\let\oldsubparagraph\subparagraph
\renewcommand{\subparagraph}[1]{\oldsubparagraph{#1}\mbox{}}
\fi

%%% Use protect on footnotes to avoid problems with footnotes in titles
\let\rmarkdownfootnote\footnote%
\def\footnote{\protect\rmarkdownfootnote}

%%% Change title format to be more compact
\usepackage{titling}

% Create subtitle command for use in maketitle
\newcommand{\subtitle}[1]{
  \posttitle{
    \begin{center}\large#1\end{center}
    }
}

\setlength{\droptitle}{-2em}

  \title{Final Assignment: Predicting Director Compensation}
    \pretitle{\vspace{\droptitle}\centering\huge}
  \posttitle{\par}
    \author{Jasper Ginn (s6100848)}
    \preauthor{\centering\large\emph}
  \postauthor{\par}
      \predate{\centering\large\emph}
  \postdate{\par}
    \date{April 8, 2019}

\usepackage{float} \usepackage{setspace} \usepackage{xcolor}
\definecolor{mypurple}{rgb}{146,76,239}
\singlespacing \floatplacement{figure}{H}

\begin{document}
\maketitle

\section{Introduction}\label{introduction}

\[
\tag{1}
\hat{\log(\text{compensation}_i)} = \beta_0 + \beta_{1} \text{age}_i + \beta_2\text{male}_i + \beta_3\text{SectorServices}_i + \beta_4\text{SectorBasicMaterials}_i + \epsilon_i
\]

From the PPC plots, we see that the data are violating the assumption of
independence as the MAP of the correlation coefficient for the observed
data indicates a strong positive correlation
\((\hat{r}_{\text{MAP, PPC}} = .466)}\).

\begin{table}[!htbp] \centering 
  \caption{Model results } 
  \label{} 
\begin{tabular}{@{\extracolsep{5pt}}lcc} 
\\[-1.8ex]\hline 
\hline \\[-1.8ex] 
 & \multicolumn{2}{c}{\textit{Dependent variable:}} \\ 
\cline{2-3} 
\\[-1.8ex] & \multicolumn{2}{c}{Compensation} \\ 
\\[-1.8ex] & (1) & (2)\\ 
\hline \\[-1.8ex] 
Constant & 4.991  (4.936, 5.046) & 4.821 (4.732, 4.910) \\ 
SectorBasic Materials &  & .225 (.079, .370) \\ 
SectorServices &  & .292 (.169, .414) \\ 
Male & .065 ($-$.085, .215) & .153 (.095, .209) \\ 
Age & .008 (0.000, .017) & .009 (.001, .017) \\ 
\hline \\[-1.8ex] 
Observations & 336 & 336 \\ 
Par. & 4 & 6 \\ 
DIC & 508 & 488 \\ 
R$^{2}$ & .021 & .098 \\ 
Residual Std. Error & .516 & .501 \\ 
\hline 
\hline \\[-1.8ex] 
\textit{Note:}  & \multicolumn{2}{r}{Baseline is sector 'Financials'} \\
\end{tabular} 
\end{table}

\subsection{Differences in Bayesian and Frequentist
inference}\label{differences-in-bayesian-and-frequentist-inference}

The effect of changing the prior variance can be summarized using a
posterior shrinking factor (CITE BETANCOURT) and posterior z-score. The
shrinking factor shows us the factor by which the variance of the
posterior distribution shrinks or expands compared to the prior
variance. The posterior z-score tells us the direction and magnitude by
which the posterior mean shifts compared to the prior mean; if we are
confident in the precision of our domain knowledge (resulting in small
prior variance), the resulting posterior weights this information
strongly and we end up with a posterior mean that lies closer to the
prior mean (represented by the z-score). However, given that we have
small variance, and if this is not corroborated by the data, then the
shrinkage factor will be large.


\end{document}
